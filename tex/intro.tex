% concept: family recipes, stuff we cook a lot, techniques/etc. we don't see often, stuff we want to pass down

\subsection{How We Cook}
We use vegetable oil by default, with butter and olive oil for flavor and extra-virgin olive oil for salads/drizzling. I habitually cut the sugar in half when baking. We never peel vegetables unless they're coated in wax, like turnips; I'd rather roast unpeeled squash and scoop out the skins than peel them. We grow basil, oregano, sage, mint, and rosemary in window baskets.

Our kitchen is equipped with:
\begin{itemize}
  \item a 10\inch\ cast iron skillet
  \item a carbon steel wok
  \item a 14\inch\ carbon steel frying pan
  \item a cast iron Dutch oven
  \item a large soup pot
  \item a huge stock / beer pot
  \item 2 medium (3qt.) and 1 small (1.5qt.) pots;
  \item a 1qt. saucier
  \item a 14\inch\ electric griddle
  \item a set of Pyrex mixing bowls
 \item 2 aluminum half-sheet pans
 \item a set of Pyrex baking dishes (loaf, 8\inch\X8\inch, 9\inch\X13\inch)
 \item a muffin tin, a 10\inch\ springform pan, a tube pan
 \item a stand mixer
 \item a food processor
 \item a stick blender w/ whisk attachment
 \item a pastry blender
 \item 3 large, 2 medium, 1 paring, and 1 bread knife
\end{itemize}

\subsection{Tips \And Techniques}

\subsubsection{how to roast veges}
Roast veges are easy, tasty, and good for you, except perhaps in high summer. The basic technique lends itself well to improvisation, so shake off your spice drawer!
  \begin{itemize}
  \item Preheat the oven to 400\0.
  \item Rinse and chop your vegetables: any root veg, cauliflower, asparagus, even broccoli.
  \item Oil a baking tray. Add the veges and drizzle with oil, salt, and pepper, or spices of your choosing. Generally I pick one flavour that complements the rest of the meal. Mix it all together (use your hands!) so veges are evenly coated.
  \item Bake for 20 min. Stir and poke to see if they're done. If not, put them in for 10-20 more min.
  \end{itemize}

\subsubsection{how to saute tofu, potatos, and veges}
Ultimately, success in sauteing comes down to knowing your stove, your pans, and your ingredients.
  \begin{itemize}
    \item Resist the urge to constantly stir---it prevents the food from browning effectively.
  \item If the pan looks dry or sounds quiet, or food is starting to stick, add more oil.
  \item Random stirring should be enough to brown food on most sides, but sometimes you will need to purposely flip pieces to avoid burning.
  \item Stick with medium or medium-high heat unless you're in a hurry.
  \end{itemize}

\subsubsection{how to maintain cast iron and carbon steel}
Many beginning cooks are nervous about cast iron, but it is fairly sturdy stuff! I encourage you to use it (for sauteing, deep frying, cornbread, but not eggs at first) as much as possible, to get a good layer of seasoning going. After each use, scrub all the food off with soap and water, dry it, then put it back on the stove with a swipe of oil from a rag or paper towel. After a few minutes, the oil should form tiny puddles, reflecting the heat pattern of the burner. Turn the stove off and let the pan cool before putting it away. Eventually, you'll be frying eggs like it's Teflon.

Carbon steel is a newcomer in our kitchen but follows the same advice as cast iron to maintain the seasoning. The pans will never look pristine after you start using them, but that's a badge of honour, not a slight on your housekeeping skills. Carbon steel is very responsive to the burner (the opposite of cast iron) and suited for stir-frying, sauteeing delicate veggies and tofu, and other foods that cook quickly. Cast iron is still the tool of choice for pancakes, eggs, and other wet, heavy items that take a lot of energy to get back up to cooking temperatures.

\subsubsection{tofu, tempeh, seitan}
(maybe move to Vegan section)
Along with legumes, these three protein sources are invaluable parts of a veg*n diet. Because they are already cooked, preparing them is about adding flavour and presentation, not food safety. East Asian cuisines have many recipes that use tofu as itself, not as a substitute; we also use tofu in place of fish or paneer in curries. Tempeh is more firm and substitutes well for pork or chicken (see Tempeh Bourguignon). \hyperref[Seitan]{Seitan} is slightly more robust than tempeh and can be molded into cutlets or sausages before cooking.

All three of these can also be simply sauteed in oil and served with a sauce. Tofu is also tasty deep-fried; if you don't care about a golden crust, consider a tasty braise (see Orange-Glazed Tofu).

\subsubsection{salad dressing, mustard, and mayonnaise}
(maybe move to Sauces section)
Did you know you can make your own salad dressing? I find store-bought dressing suspect, so we make our own. The simplest is vinaigrette, sometimes with herbs, garlic, salt, and pepper added. Use the good extra-virgin olive oil. If you've received a sampler of flavoured oils and vinegars, this is what they're for.

We also sometimes make mustard and aioli (garlic mayonnaise).

\subsection{Cookbooks/Websites We Love}
\begin{itemize}
  \item Moskowitz \And Romero, Veganomicon
 \item Bittman, How To Cook Everything Vegetarian
 \item Iyer, 660 Curries
 \item Ottolenghi, Plenty
 \item Becker et al., Joy of Cooking
 \item Julia Child, Vol. 1
 \item Brown, Tassajara
 \item Reinhart, Crust \And Crumb
 \item Katzen, The New Moosewood Cookbook
 \item Gaia’s Kitchen
 \item \href{https://www.seriouseats.com/}{Serious Eats}
 \item \href{https://smittenkitchen.com/}{Smitten Kitchen}
\end{itemize}

\subsection{How To Read These Recipes}
The title of each recipe includes a summary on how the dish is cooked. Special equipment (e.g.\ cast iron skillet, 9\inch\X13\inch\ baking dish, stand mixer) are called out. If the oven is used, the temperature and baking time are listed; in all other cases, the time given is an estimate of the active cooking time (\textbf{not} the total prep time). Temperatures are given in Fahrenheit and measurements are given in Imperial. Ingredients listed in () are optional.

I encourage you to read the entire recipe through before you start! I've tried to indicate where prep can happen in parallel but you know your own process best.

While I've tried to respect the source recipes, I've taken license to paraphrase and streamline them during transcription.
