\begin{recipe}{Vegan Nanaimo Bars}{8\inch\X8\inch\ baking dish \hfill }{\textbf{Source:} \href{https://sabotabby.dreamwidth.org/2062019.html}{Sabotabby} \hfill serves 25}
  \freeform Nanaimo bars are a Canadian institution, named after the town in BC where they were theoretically invented. They're three-layer, no-bake, and delicious. A non-vegan recipe is also given \hyperref[Nanaimo Bars]{elsewhere}.

\freeform Crust
\ing[\fr{12}]{cup}{Crisco}
\ing[\fr{14}]{cup}{sugar}
\ing[5]{tbsp.}{cocoa powder}
Melt Crisco on the stove, then stir in sugar and cocoa.

\ing[2]{}{egg replacer, hydrated}
Add the egg replacer and cook until thickened.

\ing[1]{cup}{graham crumbs}
\ing[1]{cup}{almonds or cashews, chopped}
Mix with graham crumbs and chopped nuts. Press into a 8\inch\ square pan and chill.

\freeform Filling
\ing[1]{can}{coconut milk}
\ing[1]{}{egg replacer, hydrated}
\ing[3]{tbsp.}{cornstarch}
Mix the cornstarch and egg replacer with a few spoonfuls of coconut milk in a small bowl.

\ing[2]{tbsp.}{sugar, to taste}
\ing[1]{tbsp.}{vanilla extract}
Heat the rest of the coconut milk, vanilla, and sugar on the stove until nearly boiling.

\newstep Add the cornstarch mixture and egg replacer, and heat gently until the custard thickens. Add sugar to taste.
\newstep Spread evenly over the graham-nut base, then put it back in the fridge to chill.

\freeform Topping
\ing[4]{oz.}{chocolate}
\ing[2]{tbsp.}{Crisco}
\ing[to]{taste,}{sugar}
Melt the chocolate and Crisco together; add sugar if your chocolate is unsweetened.

\newstep Let cool slightly so it doesn't melt the custard when you pour it over top. It should form an even layer. Chill until firm.

\freeform Notes:
\begin{itemize}
  \item To make these mint instead of vanilla, use 1 tsp.\ mint extract and 10--15 drops green food coloring.
  \item Traditionally, Nanaimo bars use custard powder, which is vegan, in the middle layer. Feel free to use that instead, either as stovetop custard or to flavor an American buttercream frosting.
  \item Since these are very rich and prone to squishing, cut them into bite-size pieces---5\X5 or even 6\X6 if you can manage it.
\end{itemize}
\end{recipe}