TamNonLinear’s Brownies

The original recipe posted to LiveJournal July 2nd, 2007:
http://tamnonlinear.livejournal.com/450881.html
Her revised recipe was posted April 13, 2009.
http://tamnonlinear.livejournal.com/653284.html

Timing note

There are two extended cooling times in this recipe: it is possible to make it in a day and a half if you're home a bunch, or I have made it over three days after work. Steps 0 to 18 are part 1, the rest are part 2. (Each step is not terribly long, it's the wait time, which is usually about the length of a movie or a bit more, except for the applying frosting and ganache parts.)

Before you start

Read the whole recipe first. This does require wrist and chopping strength. It involves being able to move hot pans of brownies and molten chocolate from the oven. And it uses a bunch of containers. It requires fridge space for chilling at three stages.

Basically, you are adding things to a box mix, baking, then cutting into pieces, adding a layer of frosting, and then a layer of ganache with an optional glaze step. It will make between 50 and 70 small but very dense brownies.

You will want:
- Space in your fridge sufficient to hold a pan of brownies. And then two cookie sheets of brownies.
- Containers into which to put brownies that are more like tallish pillars of chocolate. (i.e. more vertical than flat)
- A plan for dealing with washing a lot of chocolate off stuff, at least twice. Probably also washing chocolate off you. Just to warn you.

Equipment

- 1 pan, suitably sized for the brownie mix you start with (or adapt if you're okay with that)
- Two cookie sheets lined with wax paper that will fit in your fridge.
- Method of making melted chocolate (microwave bowl and microwave, double boiler, or something that can fake being one)
- Parchment paper
- Fork
- Mixing utensils
- Knife for spreading frosting
- Spoon (eating spoons work fine) for dropping ganache
- You might want measuring spoons and measuring cups, but actually, we're not so much with exact measurements of those things in this recipe.

I do a double boiler with a round mixing bowl over a pot of boiling water, it works fine.

Ingredients

- 1 box brownie mix, plus whatever is required to make it (eggs, oil, etc.)
- 3 bags of high quality chocolate chips or about 6 bars of high quality chocolate. See note below. Bags are about 12 ounces of chocolate, or substitute 2-3 bars per bag.
- 1/2 stick of butter
- 2 cups or so confectioner's sugar (powdered sugar). You'll want a little extra on hand in case.
- Pint of heavy cream (you won't use all of it.)
- Vanilla extract
- Alcohol or juice (my variant). Her original recipe uses Creme de Menthe, but Grand Marnier or raspberry liqueurs are options. You'll probably want a couple of ounces.
- Possibly other flavor extracts (i.e. orange)
- If doing the optional glaze step, you will want a jar of preserves in the appropriate flavour, fresh or fresh-frozen fruit of the same kind, and additional alcohol.

For chocolate, the original recipe used one bag white chocolate, two bags dark. Abigail had moved into picking chocolate bars that suited the flavour since it's possible to get really good dark mint chocolate or chocolate orange bars or whatever now.

Step 0

Assemble ingredients and equipment.

If you live in a country that does not stock brownie mix in the stores, you can either make yours from scratch or use something else instead. When I was in Australia I made a batch of my brownies using a chocolate mud cake mix as the base, which was a little cake-y but not bad. You'll be adding a lot of other stuff to this, so where you start isn't as vital as where you end up.

Grease the pan. All over the bottom, up the sides, and evenly. Yes, you can use spray stuff if you really want to, but I find butter works best, and if you're worried about the fat and calories, well, you probably should't even be reading this.

I make use of a parchment paper sling to make it easier to lift the brownies out when baked. If you do this, you need enough paper to lever the brownies out after baking, so leave a tail on two sides you can get a good grip on.

Somewhere between step 1 and about 5, depending on your chopping speed, you will want to pre-heat your oven to 325-350F.

Step 1

Put brownie mix in the mixing bowl.

Step 2

The brownie mix gets an additional bag of dark chips, chunks, or 2 bars of chopped chocolate bars (12 oz. total). Also, milk chocolate is for wimps and hershey's is an abomination.

Chocolate quality really does matter here. I was not as fond of the white chocolate (I substitute more dark chocolate), and I really dislike chopping chocolate, so my usual choice when making these is to chop some of it, and get good-quality chunks for the rest. There are options.

Step 3

From her original post:

Step: Preparing the white chocolate
comment: These bars about 4 ounces each. I'm using two bars, for about 8 ounces. A full bag of white chips would be even better, but they don't sell those around here anymore.

Step 4

From her original post:

Step: Chop into small pieces.
comment: A good knife is your friend

Step 5

From her original post:

Step: Add white chocolate to the mix
comment: the mix is pretty close to more chocolate than anything else already- but wait! There's more!

Step 6

From her original post:

Step: in one of the smaller bowls, mix the liquids in amounts directed from the back of the brownie mix (oil, egg, water)
comment: oil and water do mix, if you try hard enough.

ETA: I have lately taken to replacing about half the water with CdM, which doesn't seem to have resulted in any complaints yet.

From her update

Also, I'm using more booze in the recipes. I substitute most of the water in the brownie recipe with alcohol at this point. Yum!

Step 7

From her original post:

Step: add wet ingredients to the dry mix. Stir.
comment: You can develop arm strength from doing this. Don't worry about blending it perfectly, there's more stuff coming.

Step 8

From her original post:

Step:Add some vanilla essence.
comment: I don't actually measure the amount I put in. I just put as much as seems reasonable... and then double that. For this batch, I had a hard time taking the picture, so I had to use even more, because the first few attempts didn't work. See what I do for you?
If you want, you can pour in some creme de menthe at this point. I won't tell.

Also, if you are considering using imitation vanilla to save money, I ask that you please leave my journal and never return, we cannot continue to be friends.

Stir a little more- but wait! There's more!

Step 9

From her original post:

Step: Empty another bag of chips into a microwave safe bowl.
comment: You may eat one or two chips, just to make sure they are not poisonous.
Also, by the time you finish this recipe, you may be sick of chocolate for a while, so get your enjoyment in early (it's perfectly logical!).

ETA: 12/2008- I have recently started using 70% cocoa mint chocolate bars, such as valor or other high-quality brand, instead of the bags of dark chocolate chips. Even better! I use two bars for each bag of chips. I may figure out more ways to make these rich and deadly, and then the world will end.

Step 10

From her original post:

Step: Melt the chocolate
comment: Okay, here's the thing with melting chocolate... to melt evenly it has to melt slowly. This means you heat a little, stir a lot, making sure to blend it so the heat isn't hitting just one section. I give it less than a minute on high in the microwave, stir, then additional heating at 15 seconds a pop until it's just melted smooth, usually less than another minute total. You do not want to scorch the chocolate. Scorched chocolate is a waste and a cause for tears. Those lumps in the picture? Those are air bubbles I stirred in, not boiling chocolate.

My notes

If you don't have a microwave (I don't) this is time to bring out the double boiler. Put an inch or two in the bottom of one pot, fit a metal mixing bowl into to the top, simmer until the chocolate starts melting, then pour into the mixture. If you really don't feel like melting at this stage, adding more chocolate in solid form into the mixture makes a different texture, but is Still Tasty.

Step 11

From her original post:

Step: Add the melted chocolate to the brownie mix.
comment: The chocolate will start to solidify against the relatively cool mix, so stir it in while it's still soft.

Step 12

From her original post:

Step: put the brownie mix in the pan
comment: Do not even attempt to pour it out of the bowl. This stuff does not pour, and the bowl is going to be *heavy*. Use a good, solid serving spoon and scoop it in there. Use a rubber scraper to get the last of it and then flatten it out in the pan.

Step 13

From her original post:

Step: Put brownies in the oven
comment: Oven is preheated to about 325-350 degrees (mine leaks heat, so temperature is a guess).

Step 14

From her original post:

Step: Bake
comment: for about an hour. The brownie mix says 45 minutes, but I find that leaves the brownies a little goopy. A little more time won't hurt, though you want to avoid burning.

I usually start washing up the bowls and pans at this point. That can take a while.

My notes

I have had more issues with burning than gloopy, so monitoring is probably your friend here. (I'd start with what the mix says, adapt if you changed pan size slightly, and check regularly.)

Step 15

From her original post:

Step: Remove brownies from oven
comment: I swear they get heavier in there.

Please note, if you're use to doing a 'stick test' to see if a pan of brownies is done (stick in a toothpick, see if it comes out clean), don't even bother. These are molten chocolate at this point. You just have to have faith that they are done. Besides, we're about to do something fun

Step 16

From her original post:

Step: Measure out some creme de menthe
comment: A purely medicinal amount, really.

Important step of which I did not get a picture: Pour the creme de menthe over the fresh-from-the-oven-hot brownies. I mean RIGHT NOW. It sizzles.

I swear this cooks off all the alcohol. Really.

From her update

This is just more of a general tip, but I find that the post-baking alcohol soaks in a lot better if I stab the hot-from-the-oven brownies with a fork before I pour on the booze. It helps prevent having it all puddle in one area once it's poured on.

My notes

If you are not using alcohol, a similar amount of non-pulp juice works here, or a smaller amount of flavoured extract. I've done pomegranate and cherry juice at different times, or orange extract.

Step 17

From her original post:

Step: Mix creme de menthe into the hot brownies.
comment: QUICK! Get a fork and start stabbing the brownies all over, so the creme de menthe soaks in evenly. You don't quite want to stir the mix again, but be generous with your stabbination. You can feel free to quote Moby Dick, if you like.

It may look a little goopy at this point, but it's probably okay.

If you suspect it is goopier than it should be, you can put it back in the oven (turn the oven off, first) for ten minutes, just so the additional heat dries it out a bit.

I have, on occasion, turned off the oven and accidentally left the brownies in there overnight, because I am hideously absentminded, without it resulting in burned brownies.

Important step of which I did not get a picture: Now leave it alone and let it cool for a while. Once it's reached room temperature, cover with plastic wrap or aluminum foil.

Step 18

From her original post:

Step: Chill brownies overnight
comment: Easiest part of the recipe, this only requires the passage of time, which needs no effort on your part.

And the Very Next Day... (Or later, I don't really care)

Step 19

About 30-60 minutes before you want to start doing things to the next stage of brownies, take a stick of butter out of the fridge and let it come to room temperature. You might want to give it longer if your kitchen is cool.

Step 20

From her original post:

Step: Get the brownie block loose in the pan.
comment: You need to get the brownies out of the pan. I recommend a warm water bath to loosen the block of brownies, obtained by placing the pan in a larger pan, and then putting hot tap water in the larger pan (but not on the brownies).

Magic tip: put a pair of butter knives underneath to elevate the bottom of the pan and allow water flow under it. Use a knife around the edge of the brownie pan to get the sides loose. If you're lucky, the block will start shifting when you're doing this.

If you're dexterous, you can pick the brownie pan up and lift it overhead, to see if the butter on the bottom of the pan has melted. Don't mind the water droplets, there's no dignity in this.

eta: also, another suggestion on getting around this problem is in the comments. Check it out.

My notes

The other suggestion is the parchment sling, previously described.

Step 21

From her original post:

Step: Place cutting board over top of the pan
comment: This is why it's good to wash up as you go.

Step 22

From her original post:

Step: Invert! Watch block of brownies slide down!
comment: If you know they're loose, you can just wait and watch as they release their grip on the bottom-now-top of the pan and fall.

If you have unresolved aggression, you can also try thumping the container, or even picking up the whole assembly and whacking it against the edge of the counter, but this should ONLY be attempted if you are entirely certain that the brownie block is solid throughout.

Step 23

From her original post:

Step: Remove pan, view block of brownies.
comment: Pat dry with paper towel to remove excess butter.

Step 24

From her original post:

Step: Square off the block of brownies
comment: This requires a long, sharp knife and some wrist strength. Even out the edges so the brownie bits you cut off later don't end up wobbly.

My notes

If you have burned your edges as I am wont to do, this may be a place to discard them, rather than use them as in the next steps. Just so you know. The rest of this will go easier if you trim to not-burned-hard bits. If your brownies just seem dense and a bit dry, don't worry, the next major step will help.

Step 25

From her original post:

Step: Save the edges!
comment: Waste not, want not. We'll be using these later.

Step 26

From her original post:

Step: Cut a strip off!
comment: Once you have the solid block of brownies, cut a strip off along one side. I'd say you want something maybe 1/2 inch wide. Again, a long, sharp knife and some wrist strength is useful here. You have to just push downward, no sawing allowed.

Step 27

From her original post:

Important step of which I did not get a picture: Now, turn the knife and cut a small square off the end of the strip of brownie.

You might want to ask important questions such as if you have your orientation correct and your size reasonable. I'm here to tell you, it doesn't matter and I don't judge. As long as you end up with a small blockish brownie thing that is vaguely stable and will eventually fit inside a container, it's good.

I will say, wide enough to be stable, small enough to be edible are the parameters.

I find that the brownie blocks, after the addition of icing and ganache, are often just a little bit too tall for standard containers, so I often make thinner cuts and turn them on their side if I know I'm adding a substantial amount on top later. This is also a solution to wobbly brownies.

Important step of which I did not get a picture: Also, you want to have two cookie sheets lined with wax paper handy. This is where your little blocks will go. You have read the entire recipe before starting, right?

My notes

I agree with 'thinner cut, turn on side' advice here, and just generally find them easier to cut.

Step 28

From her original post:

Step: The block cut up into little cubes
comment: Aren't they cute? Also, you can scrape the crumbled bits left on the cutting board into the same container as the edge strips you saved earlier.

Important step of which I did not get a picture: Now we're going to start making icing. You will need butter and confectioner's sugar in amounts I have never quite nailed down. Let's say a half stick of butter and two cups of sugar. Adjust as feels vaguely appropriate. I make this up every time.

Butter should be room temperature or softer (room temperature in December is not room temperature in August, at least in my house). You can gently zap it in the microwave and blend it smooth if a bit of it melts.

My notes

As I said, you wanted that room temp butter!

Step 29

From her original post:

Step: here's the Sugar
comment: The butter should already be in a bowl, soft, and ready to be blended.

Step 30

From her original post:

Step: Blend sugar and butter
comment: Don't worry if it still looks dry, just try to get the worst of the lumps out.

My notes

If you are nervous about making buttercream frosting, here is a recipe from The Kitchn and here is one from the Smitten Kitchen. (Ignore the cream/milk ingredient: we are using other liquid, but you might feel better with photos of the process.)

Step 31

From her original post:

Step: More Creme De Menthe!
comment: makes it smooth and colorful. You will be surprised how quickly the sugar dissolves into the booze. Or at least I was, the first time. You might be harder to impress.

My notes

You can use alternate liquids here! If you want to do non-alcohol versions, juice of the same kind you mixed into the brownies works, or orange or mint extract. You can also add a little vanilla extract here, or just vanilla. Many flavour combinations are possible. Note that juices will also produce colour changes.

Step 32

From her original post:

Step: blend!
comment: If something bright green and full of alcohol can ever be said to blend...

Step 33

From her original post:

Step: Top brownies with icing
comment: Icing should be smooth enough to spread easily or apply in dollops. If not smooth enough, warm (carefully!) or add more CdM.

From her update:

Next stage: When applying the icing, IF this is a recipe that includes a glaze layer, it's very helpful to form a little well on top of the brownie. Icing shapes fairly nicely.

If I was any good with a pastry bag I'd pipe it on in a design, but my one experiment with this resulted in the bag exploding in my hands and getting icing on one of the cats, so I'm hesitant to try a repeat.

My notes

I normally do this with a butter knife, and smear a little on, spread it around, rinse and repeat. If you are worried your brownies are too dry and dense, the frosting will seep into them and make them softer as they sit.

Step 34

From her original post:

Step: Chill frosting
comment: Put the brownies in the fridge to cool until icing is firm and solid. Takes a little bit. Go watch a movie!

Wait, one little thing to do first

Step 35

From her original post:

Step: soak the edge bits
comment: You remember all those little edge bits and crumbled off things? Take a spoon or a fork and break them down as well as you can into brownie dust. Pour enough heavy cream over to just cover, and set the bowl aside (in the fridge) to let them soak and soften.
Now you should go watch a movie.

My notes

This is the step I usually skip, because I had burned bits, not crumbly bits. It still works fine without it.

Optional glaze step 35

This is an addition from her second post, where she says:

Ingredients:
A jar of preserves in the flavor you're trying to capture. Pick something of good quality, without high frustose corn syrup (bleah!) and without too many other flavors in it (I do this for the raspberry recipe as well, and have found that preserves with grape tend to taste like, well, grape rather than raspberry).

Fresh (or fresh frozen) fruit of the same variety (I'm using blood oranges here, because I like the color)

Booze (duh)

Empty the jar into a microwave safe container. Add the fresh fruit. Heat it up and stir a lot, until the preserves are liquid.

Add a little booze. Can't hurt.

Strain!

chances are that the preserves have seeds or rind or something else, and the fresh fruit likely does as well. This is why we heat it up, so that it's liquid enough to go through a strainer to remove all that stuff.

Reduce!

Now you need to boil your glaze down until it's thick enough to form a layer when cooled.

I'm sure there's some better way to do this, but I just boil it and stir a lot, and assume it's done when the mixture is thick enough that it parts behind the spoon as I stir.

Let it cool a little before applying it to the brownies. You want it just cool enough that it doesn't run off like water, and still warm enough that it isn't forming clumps. Warm it up a little more if it is clumping.

And hey, add some booze. A lot of it probably cooked off earlier, right?

Now pour it on the brownies, a spoonful at a time. This is why we built those little divots into the icing layer. Pretty, no?

My notes

This is also a step I have never done, because I mostly don't trust myself with large pots of hot sugar liquid. Variations, like a smear of preserves without the rest of it would also be tasty.

You will want to chill it again after this step and watch another movie.

She says:

Good movie?

and now, ganache!

Step 36

From her original post:

Step: Add chips to the brownie bits
comment: Take your last bag of chips and add to the cream-soaked brownie bits. Add enough additional cream to cover.

Ganache is generally a ratio of anywhere from 1:1 to 1:3 of cream and chocolate. Heavier on the cream and it is softer, heavier on the chocolate and it is more shell-like.

I'm sure you'd be surprised to know that I don't much measure, myself.

If you're worried, go lighter on the cream. Remember, if it looks too thick, you can add more cream later. If you've made it too liquid, you can't add less. Although this may be a good reason to keep even more chocolate around. It's your decision.

My notes

Again, if you don't have brownie bits, don't worry about that and just go for the mix of cream and chocolate. I promise it is still tasty. I usually go for about a 1:2 ratio, but I eyeball a bit.

Step 37

From her original post:

Step: Heat, melt, blend
comment: The same rule applies here as before with melting chocolate: heat a little, stir a lot. You do not want to scorch the cream or the chocolate, and chocolate needs time to melt. Stir, stir, stir, and suddenly it starts looking like thick pudding. Yay!

My notes

As previously, I do this in a double boiler.

Step 38

From her original post:

Step: Add more creme de menthe
comment: Just a splash. I swear this is just because the CdM makes the ganache shiny and smooth. Really. Trust me.

My notes

If not using alcohol, again, you can add a dash of whatever makes sense. (Be careful about adding too much liquid if it's juice, but a tiny bit is fine.) Flavour extracts are also fine.

Step 39

From her original post:

Step:: Apply ganache to the brownies.
comment: Dollop. Dollop. Dollop nudge nudge dollop.

If the ganache gets lumpy or stops flowing, heat it again a little.

My notes

I make my ganache to 'barely spoonable' and do a spoonful over each, enough to cover most of the frosting, but hopefully not drop down too much onto the wax paper on the cookie sheet.

Step 40

From her original post:

Step: Everybody gets some ganache.
comment: If you have more ganache left over after the first distribution, rotate the trays, evaluate from another angle, and figure out where you can add more.
(oddly enough, this reminds me of playing SimCity.)

Step 41

From her original post:

Step: Back in the fridge to let the ganache set.
comment: Go watch another movie. This is why I like netflix.

Step 42

From her original post:

Step: Wait until the ganache has set
comment: Cooled ganache should be firm. Please keep the poking of food to a minimum.

Step 43

From her original post:

Step: Put brownies in containers
comment: Leave enough room to be able to pry them out again later.

Step 44

From her original post:

Step: Final count!
Comment: Somewhere in the neighborhood of 50 brownies.

Final step

From her original post:

Important step of which I did not get a picture: Washing up. This included the counters, the dishes, the utensils, the bowls, and the cook. I end up with chocolate on my elbows and my ears from this. No, I don't know how. Also, I usually don't want anything chocolately for several days after.

Notes

From her original post:

These will keep a while (keep refrigerated). Most people can manage one a day or less. More than two and you start to lose sleep. More than three and you'll see things.

Please feel free to use to bribe people.

My final notes

I've done my best to condense the ingredient notes and to make clarifying comments where it's helpful or the photographs were informative and the text was a bit confusing. If you think I've missed something, please let me know.
