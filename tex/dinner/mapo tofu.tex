\begin{recipe}{Mapo Tofu}{wok \hfill }{\textbf{Source:} The Food of Sichuan \hfill serves 4}
  \ing[14]{oz.}{tofu (1 block), cubed}
  Steep tofu cubes in very hot, lightly salted water while you prep other ingredients.

\ing[5]{tbsp.}{oil}
\ing[2\fr{12}]{tbsp.}{Sichuan chili bean paste}
Heat a wok on medium. Add oil and chili bean paste and stir fry until oil has turned red.

\ing[1]{tbsp.}{fermented black beans}
\ing[2]{tsp.}{ground chilis}
\ing[1]{tbsp.}{garlic, finely chopped}
\ing[1]{tbsp.}{ginger, finely chopped}
Add the black beans and ground chilis and stir fry for a few seconds until they are fragrant. Add the ginger and garlic and ditto. Turn down the heat if the aromatics are browning too quickly.

\ing[\fr{34}]{cup}{stock or water}
\ing[\fr{14}]{tsp.}{white pepper}
Remove tofu from its brine, shaking off excess moisture. Add it directly to the pan, along with the liquid and pepper. Stir gently to coat the tofu cubes without breaking them.

\ing[1]{tbsp.}{potato or corn starch}
\ing[2\fr{12}]{tbsp.}{cold water}
Bring to the boil, then simmer on low for a few minutes. In the meantime, mix the starch with the water. Add it one spoonful at a time, stirring gently as the sauce thickens. Do not over-thicken---the sauce should cling gently to the tofu.

\ing[2]{}{scallions}
\ing{}{Sichuan pepper}
Cut the scallions to \fr{34}\inch\ lengths. Roast and grind the Sichuan pepper. Add them to the wok, stir to heat through, and serve.

  \freeform Notes:
  \begin{itemize}
    \item Traditionally this recipe is made with ground beef or pork in addition to the tofu. Dunlop feels the meat can be skipped without loss of deliciousness, but if you are accustomed to the meat garnish, Bittman suggests fried crumbled tempeh.
  \end{itemize}
\end{recipe}