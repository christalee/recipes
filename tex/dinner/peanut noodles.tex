\begin{recipe}{Peanut Noodles}% \hfill %temperature \hfill %time}
 {\textbf{Source:} \textit{The Enchanted Broccoli Forest}, by way of \textit{Gaia's Kitchen} \hfill serves 6--8}

 \freeform This is one of those recipes that has diverged so much from its published form, I just had to write down our preferred way.

 \ing[1\fr13]{cup}{peanut butter}
 \ing[1]{cup}{water}
 \ing[4]{tbsp.}{cider vinegar}
 \ing[6]{tbsp.}{soy sauce}
 \ing[6]{tbsp.}{molasses}
 Heat sauce ingredients over medium heat, stirring constantly, until they form a smooth mixture. Bring to a boil, then simmer gently until sauce thickens and darkens. Set aside.

 \ing[1]{lb.}{extra firm tofu}
 Cut the tofu into cubes and fry until golden brown on at least 3 sides. Dump into a large serving bowl and mix with the peanut sauce.

 \ing[2]{tbsp.}{ginger, chopped}
 \ing[4]{cloves}{garlic, chopped}
 \ing[1]{}{onion, sliced}
 Saute the aromatics after the tofu, then add to the bowl of fried tofu.

 \ing[1]{lb.}{pasta}
 Boil pasta as directed in salted water. (Steam veggies over top if desired.)

 \ing[4--6]{cups}{veggies, trimmed}
 Stir-fry, roast, or steam the vegetables: carrots, peppers, broccoli are perennial favorites; or try cauliflower, snap peas, green beans, mushrooms, eggplant, or any leftover veggies in the fridge.

 \newstep Add the vegetables and noodles to the saucy tofu. Mix carefully but thoroughly. Serve, optionally garnished with scallions and/or chopped peanuts.
\end{recipe}
