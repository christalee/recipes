\begin{recipe}{Potato-Chickpea Curry}{pressure cooker \hfill stovetop \hfill 30 min.}{\textbf{Source:} Daniel Bergey \hfill serves 4}
  \freeform A classic, hearty dish for wintertime.

\ing[1]{cup}{chickpeas, dry}
Soak chickpeas 4-6 hr. Pressure cook 30 min.

\ing[1]{}{onion}
\ing[3]{}{potatos}
Chop onion and cube potatos. Sweat onion in wok until translucent. Add potatos and fry 10-15 min., until slightly browned but still quite firm.

\ing[\fr{12}]{cup}{water}
Add water, cover, and simmer over low heat for ~5 min. Add water or uncover as needed, to make sure the potatos cook through and the pan is dry by the end of this time.

\ing[1]{tbsp.}{cumin seeds}
\ing[1]{tbsp.}{coriander seeds}
\ing[1]{tsp.}{red pepper flakes}
\ing[1]{tsp.}{ground turmeric}
Add cooked chickpeas and spices to the wok. Simmer 5 min. to meld flavors. Potatos should start to break down and make a thick "sauce".

\freeform Notes:
\begin{itemize}
  \item It's convenient to double this recipe, or cook more than 1 cup of chickpeas at a time, and use the rest for another dish (like hummus!)
  \item I frequently add frozen peas at the end of this recipe, for color contrast and a pop of sweetness.
\end{itemize}
\end{recipe}