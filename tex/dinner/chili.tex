\begin{recipe}{Veggie Chili}{pressure cooker \hfill stovetop \hfill 1.5 hrs.}{serves 8}

 \freeform This recipe is very flexible, so use a mix of beans and vegetables that suits you! As written it is fairly mild.

 \ing[2--3]{cups} {dry beans: black, kidney, pinto, white, garbanzo}
 \ing[6--8]{cups}{water}
 \ing[2]{tbsp.}{salt}
 \ing[2]{}{bay leaves}
 \ing[4]{}{dried red peppers}
 Start beans in the pressure cooker with aromatics. While they cook, chop vegetables, bake cornbread, and/or start rice cooking.

 \ing[1--2]{tbsp.}{cumin}
 \ing[1--2]{tbsp.}{coriander}
 \ing{1--2}{dried red peppers or flakes}
 Optional: Toast whole spices in a skillet. When they smell good, grind them in the mortar.

 \ing[2]{}{onions, peeled \And chopped}
 \ing[3]{}{carrots}
 \ing[6--10]{cloves}{garlic, peeled \And chopped}
 \ing[1]{}{green pepper}
 Saute onions, carrots, and spices (untoasted and whole) in a skillet. When the onion is brown, add the garlic, green peppers, and spices (toasted and ground).

 \ing[\fr{12}]{cup}{wine or beer}
 \ing[28]{oz.}{diced tomatoes (any flavor)}
 \ing[2]{tbsp.}{oregano}
 \ing[2]{tbsp.}{basil}
 \ing[1]{can}{(water chestnuts)}
 When vegetables have started to soften, add the skillet contents, water chestnuts, herbs, and canned tomatoes to the beans. Deglaze the skillet with wine, beer, or tomato juice and add the results to the beans. Stir and return to a simmer.

 \ing[1]{cup}{(TVP)}
 Simmer the chili until the flavors and consistency are correct or your patience runs out. Add wine, water, or stock if it's too dry; add TVP or boil down if it's too wet. Adjust seasonings with more of what's missing: salt, tomato paste, hot sauce / chipotle, cocoa powder\dots

 \newstep Serve chili in bowls, with grated cheese, yogurt, cornbread, and/or rice.

 \freeform Notes:
 \begin{itemize}
  \item 1--2 cups frozen corn can go in any time. Kale or other greens should go in 5--10 minutes before serving.
  \item If you want to use canned beans, aim for 6 cups cooked beans total---3 large cans should do it.
 \end{itemize}
\end{recipe}
