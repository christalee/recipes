
\documentclass{article}
\usepackage{cb_recipe}

\begin{document}
\begin{recipe}{Apple Matzo Charlotte}{serves 12}{350\0 F \hfill 9"\X13" baking dish \hfill 45 min.}
  % !tags: dessert, Passover, recipe

  \freeform A perennial favorite at Passover.
  \textbf{Source:} Adapted from \textit{Jewish Cookery}, Matzo Charlotte \#2
  \freeform\rule{\textwidth}{0.05pt}

  \ing[4]{}{matzos}
  Soak matzos in water until soft. Drain and squeeze out excess water.
  \freeform\rule{\textwidth}{0.05pt}

  \ing[6]{}{eggs, separated}
  \ing[\fr14] {cup} {sugar}
  \ing[4]{}{apples, sliced}
  \ing[\fr12] {cup} {(almonds)}
  \ing[\fr12] {cup} {(raisins)}
  \ing[1] {tbsp} {cinnamon}
  Mix matzos with beaten egg yolks, sugar, apples, nuts, raisins, and cinnamon.
  \freeform\rule{\textwidth}{0.05pt}

  \ing[\fr14] {cup} {butter}
  Add butter, melted or chopped into small pieces.
  \freeform\rule{\textwidth}{0.05pt}

  \ing[6] {}{egg whites}
  Whip egg whites until stiff. Fold gently into apple-matzo mixture.
  \freeform\rule{\textwidth}{0.05pt}

  \newstep Bake in a well-greased casserole until golden brown on top.

  \freeform Notes:
  \begin{itemize}
    \item The original recipe suggests this can be served plain or ``with lemon sauce", whatever that is.
    \item Chop the apples while the matzos soak in the baking dish; dry and grease the baking dish later, while the egg whites are whipping.
  \end{itemize}
\end{recipe}
\end{document}
