
\begin{recipe}{Veggie Chili}{serves 8}{stovetop \hfill pressure cooker \hfill 1.5 hrs.}

 \freeform This recipe is very flexible, so use a mix of beans and vegetables that suits you! As written it is fairly mild.

 \Ing{1 cup each: black, kidney, white, garbanzo beans (pick 2--3)}
 Start beans in the pressure cooker. While they cook, chop vegetables, bake cornbread, and/or start rice cooking.

 \Ing{1--2 tbsp.\ each cumin, coriander}
 \Ing{dried red pepper, to taste}
 Optional: Toast whole spices in a skillet. When they smell good, grind them in the mortar.

 \Ing{2 onions, peeled and chopped}
 \Ing{6--10 cloves of garlic, peeled and chopped}
 \Ing{3 carrots}
 \Ing{1 green pepper}
 Saute onions, garlic, carrots, and spices (if not previously toasted) in a skillet. When the onion is brown, add the green peppers.

 \Ing{\fr{12} cup wine or beer}
 \Ing{1--2 small cans diced tomatos (any flavor)}
 \Ing{2 tbsp.\ each oregano, basil}
 \Ing{(1 cup TVP)}
 When vegetables have started to soften, add the skillet contents, herbs, and canned tomatos to the beans. Deglaze the skillet with wine, beer, or tomato juice and add the results to the beans. Stir and return to a simmer.

 \newstep Simmer the chili until the flavors and consistency are correct or your patience runs out. Add wine, water, or stock if it's too dry; add TVP or boil down if it's too wet. Salt, then add more of whatever's needed: tomato paste, hot sauce / chipotle, cocoa powder\dots

 \newstep Serve chili in bowls, with grated cheese, yogurt, cornbread, and/or rice.

 \freeform Notes:
 \begin{itemize}
  \item 1--2 cups frozen corn can go in any time. Kale or other greens should go in 5--10 minutes before serving.
  \item If you want to use canned beans, aim for 6 cups cooked beans total---3 large cans should do it.
 \end{itemize}
\end{recipe}
