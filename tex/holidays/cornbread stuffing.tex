\begin{recipe}{Cornbread Stuffing}{9\inch\X13\inch\ baking dish \hfill 375\0F \hfill 45 min.}{\textbf{Source:} adapted from \href{https://www.seriouseats.com/cornbread-stuffing-sausage-sage-recipe}{Serious Eats} \hfill serves 8}
  \freeform Once upon a time, my family checked a bunch of cookbooks out of the library and picked recipes we wanted to try for the holidays. I picked cornbread stuffing and have never looked back.

\ing[\fr{12}]{recipe}{\href{https://www.seriouseats.com/southern-unsweetened-cornbread-recipe}{cornbread}}
Heat the oven to 425\0F. Cut the cornbread into \fr{34}\inch\ cubes and spread onto a baking tray. Toast for 10 min. Set aside to cool.

\ing[4]{tbsp.}{butter}
\ing[3]{}{vege sausages}
Brown the sausages, then remove them from the pan. Slice them into rounds or chop more finely, as desired. Return them to the pan to brown.

\ing[1]{med.}{onion, chopped}
\ing[4]{cloves}{garlic, chopped}
\ing[1]{}{bell pepper}
\ing[8]{oz.}{mushrooms}
\ing[2]{tbsp.}{fresh sage, chopped}
Add vegetables and herbs to the pot and cook, stirring frequently, until vegetables are starting to brown, about 10 min.

\ing[\fr{12}]{cup}{vegetable stock}
Remove from heat and add stock, stirring to deglaze. Salt and pepper if you haven't already.

\ing[1]{cups}{vegetable stock}
\ing[2]{}{eggs}
\ing[2]{tbsp.}{parsley, chopped}
Whisk stock and eggs and parsley together. Add them to sausage mixture, stirring to prevent the egg from cooking. Add the cornbread cubes and stir gently but thoroughly.

\newstep Scoop the stuffing into a greased baking dish. Bake at 375\0F until top is browned and interior has reached 150\0F. Let cool 5 min. and serve.

\freeform Notes:
\begin{itemize}
  \item We typically use Field Roast Apple-Sage vege sausages, but if you have a good breakfast/uncased vege sausage you like, go ahead and use it here. TVP crumbles might also do in a pinch.
  \item The eggs are easily skipped, if you want to make this vegan. (Of course, you have to make your cornbread vegan too.)
  \item Unsweetened cornbread is the right kind to use here, not the sweet cake-y stuff.
\end{itemize}
\end{recipe}