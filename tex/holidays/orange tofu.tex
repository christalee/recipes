\begin{recipe}{Orange-Glazed Tofu}{9\inch\X9\inch baking dish \hfill 350\0F \hfill 1 hr.}{\textbf{Source:} inspired by \href{https://web.archive.org/web/20101121022851/https://www.theppk.com/2010/04/orange-ginger-baked-tofu/}{PPK's Orange Ginger Baked Tofu} \hfill serves 4}
 \freeform Easy, tasty, hands-off, suitable for hanging out in the oven indefinitely - this was our first go-to vegetarian holiday centerpiece.

 \ing[\fr12]{can}{orange juice concentrate}
 \ing[\fr14]{cup}{molasses}
 \ing[\fr14]{cup}{soy sauce}
 \ing[2]{tbsp.}{sesame or olive oil}
 \ing[2]{tbsp.}{ginger, chopped}
 \ing[3]{cloves}{garlic, chopped}
 \ing[1]{tsp.}{crushed red pepper}
 Mix marinade ingredients together in a bowl.

 \ing[1]{block}{tofu}
 If this is a centerpiece: keep the tofu in a single brick. Otherwise, slice or cube tofu to your preference. Place tofu in an oiled baking dish and pour marinade over top.

 \newstep Marinade the tofu at least 30 min., then bake until marinade has thickened to a sauce and tofu edges have browned, at least 1 hr. If you have time, baste the tofu with the sauce every 20--30 min. so the top stays moist.

 \freeform Notes:
 \begin{itemize}
  \item Tofu can be baked quicker at 400\0F or even 450\0F; once baked, it can be held at 200\0F until the rest of the feast is ready.
  \item The marinade/sauce can also be made on the stovetop, optionally thickened with 1 tbsp. cornstarch.
  \item Tofu texture can be improved by frying or broiling the tofu before adding the sauce, or possibly even freezing. But the simple dump-and-bake version is quite respectable.
  \item This recipe appears to be superseded by \href{https://www.theppk.com/2016/12/sweet-smoky-glazed-tofu-ham/}{Sweet \& Smoky Glazed Tofu Ham}, which looks cute but mostly leads to burnt orange slices.
 \end{itemize}
\end{recipe}
