\begin{recipe}{Apple Matzo Charlotte}{9\inch\X13\inch\ baking dish \hfill 350\0 F \hfill 45 min.}{\textbf{Source:} adapted from Leah Leonard, \textit{Jewish Cookery}, Matzo Charlotte \#2 \hfill serves 12}
 % !tags: dessert, Passover, recipe

 \freeform A perennial favorite at Passover.

 \ing[4]{}{matzos}
 Soak matzos in water until soft. Drain and squeeze out excess water.

 \ing[6]{}{eggs, separated}
 \ing[\fr14] {cup} {sugar}
 \ing[4]{}{apples, sliced}
 \ing[\fr12] {cup} {(almonds)}
 \ing[\fr12] {cup} {(raisins)}
 \ing[1] {tbsp} {cinnamon}
 Mix matzos with beaten egg yolks, sugar, apples, nuts, raisins, and cinnamon.

 \ing[\fr14] {cup} {butter}
 Add butter, melted or chopped into small pieces.

 \ing[6] {}{egg whites}
 Whip egg whites until stiff. Fold gently into apple-matzo mixture.

 \newstep Bake in a well-greased casserole until golden brown on top.

 \freeform Notes:
 \begin{itemize}
  \item The original recipe suggests this can be served plain or ``with lemon sauce,'' whatever that is.
  \item Chop the apples while the matzos soak in the baking dish; dry and grease the baking dish later, while the egg whites are whipping.
  \item This is presumably a Passover version of the classic English pudding, \href{https://www.theguardian.com/food/2020/oct/21/how-to-make-the-perfect-apple-charlotte-recipe}{Apple Charlotte}.
 \end{itemize}
\end{recipe}
