\subsection{how we veganize things}
Daniel was vegan for many years, and we still have friends and occasions where dairy, eggs, and honey are prohibited. We don't keep too many substitute ingredients around, so these recipes mostly use stuff you probably have in your cupboard. To convert any recipe to be vegan, identify the basic elements of the dish (fat, protein, acid, gluten, sugar, moisture) and make sure your substitutes are roughly similar.

\subsubsection{umami ("meat")}
Many vegans swear by nutritional yeast for adding depth of flavor, but it can be easy to over-do, and I don't find it makes convincing cheese sauce. YMMV. Soy sauce, miso, and mushrooms fit into many recipes seamlessly. Bragg's liquid aminos and vegan worcestershire sauce (traditionally contains anchovies) are also worth experimenting with, if you can find them.

\subsubsection{bacon}
This gets a special mention, given its hallowed place in mainstream American cooking. Bacon is basically salty, smoky, chewy bits. Replace it with liquid smoke, smoked paprika, or chipotle powder and move on.

\subsubsection{butter}
Crisco is more multi-purpose than coconut oil. Vegan pies for everyone! Vegan Earth Balance (not all flavors are!) is also great for biscuits or just spreading on bread. In baking, nut butters (peanut, almond, sunflower, cashew, tahini) and avocados can provide creamy binding power.

\subsubsection{eggs}
Once in a while you just need to bake a cake, and egg replacer is great. Don't try to make a frittata, though - stick with silken tofu for that. Extra-firm tofu is best for scrambling. Apple sauce is often recommended to provide sugar/egg/moisture in vegan baking, but I never have it around; if you use it, consider thinning it with water unless it's replacing yogurt.

\subsubsection{buttermilk}
I never have buttermilk in my fridge, so I usually add 1 tbsp. lemon or apple cider vinegar to 1 cup milk instead. The acid reacts with the baking powder/soda to leaven your baked goods. Some recipes might enjoy balsamic or wine vinegar - use your judgment!

\subsubsection{milk}
One summer, our food group stocked dairy and non-dairy milk, and I had a chance to compare soy, rice, and almond milk. I like soy milk for general purposes, but rice milk is best on cereal. Almond and coconut milk can be pricey but are definitely delicious, so look out for sales and/or ways to buy in bulk! I don't really recommend non-dairy yogurt, cheese, etc. but I've heard good things about Daiya vegan cheese.

\subsubsection{bechamel}
This relies on a roux, made with oil and non-dairy milk. Otherwise it's a classic bechamel, that you can take in any flavor direction.

## Max's tofu scramble

## vegan Nanaimo bars
Nanaimo bars are a Canadian institution, named after the town in BC where they were theoretically invented. They're three layer, no-bake, and delicious.
http://sabotabby.livejournal.com/1344653.html

#### Layer 1
1/2 cup Crisco
1/4 cup sugar
5 Tbsp cocoa powder
2 egg replacer **hydrated?
1 cup graham crumbs
1 cup chopped almonds or cashews

Melt Crisco on the stove, then stir in sugar and cocoa. Add the egg replacer and cook until thickened. Mix with graham crumbs and chopped nuts. Press into a 8" square pan and chill.

#### Layer 2
1 can coconut milk
3 Tbsp cornstarch
1 egg replacer **hydrated?
3 Tbsp sugar
1 Tbsp vanilla extract

Mix the cornstarch with a few spoonfuls of coconut milk in a small bowl. Heat the rest of the coconut milk, vanilla, and 2 Tbsp sugar on the stove until nearly boiling. Add the cornstarch mixture and egg replacer, and heat gently until the custard thickens. Add sugar to taste. Spread evenly over the graham-nut base, then put it back in the fridge to chill.

Mint Nanaimo bars: instead of vanilla, use mint extract and green food coloring.

**Or just use custard powder??

#### Layer 3

4 oz. chocolate
2 Tbsp Crisco
sugar to taste

Melt the chocolate and Crisco together and add sugar to taste. Let cool a bit so it doesn't melt the custard too much when you pour it over top. It should form an even layer. Chill until firm.

## vegan millionaires
http://sabotabby.livejournal.com/1372333.html

## vegan latkes
When we met, Daniel was vegan. This recipe dates from that era, and has been highly appreciated by gluten-free guests as well as anyone who favors simplicity.

potatos
1 onion
oil for frying

Cut up 1/4 of your potatoes and boil, then mash. Meanwhile, shred the potatos and onion with a box grater or food processor. Optional: lightly salt and drain in a colander, pressing to remove as much moisture as possible.

Once the mashed potatos are cool enough to handle, mix them into the shredded potatos, so they hold together in patties. Fry the latkes in a well-oiled cast iron or non-stick pan, on the lowest heat you have patience for.

Check for doneness by tasting a piece (like any pancake). Hold finished latkes on a plate lined with paper towels, maybe in the oven, but they're best fresh. Serve with apple sauce, sour cream, and/or sauerkraut.

## avocado pie from Max??
## vegan matzo balls?

## Molly's coconut whipped cream
This is a technique more than a recipe. It relies on the cream at the top of a can of coconut milk, so if you usually get a cheap low-fat brand, spring for the good stuff. It's tough to get right - don't despair, just mix it back into the coconut milk and make pina coladas.

canned coconut milk (NOT coconut cream)

Freeze the cans for an hour (!?), long enough to get cold but not so long that it freezes solid and bursts the can. Also chill a mixing bowl and beaters in the fridge or freezer.

Open the can and spoon out the solid white coconut fat, leaving the clear coconut milk in the can for another recipe. Whip the coconut cream until it forms soft peaks, adding sugar near the end. If it looks melty, put everything in the fridge to cool off.

Serve immediately, with cake.

## vegan chocolate cake
